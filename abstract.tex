\noindent {The performance of most error-correction algorithms that operate on genomic sequencer reads is dependent on the proper choice of its configuration parameters, such as the value of $k$ in \textit{k}-mer based techniques.} In this work, we target the problem of finding the best values of these configuration parameters to optimize error correction.
We perform this in a data-driven manner, due to the observation that different configuration parameters are optimal for different datasets, \ie different instruments and organisms. 
%, and more importantly, downstream genome assembly. %have not used genome assembly now so misleading/removed (SC10302018)
We use language modeling techniques from the Natural Language Processing (NLP) domain in our algorithmic suite, \textsc{Athena}, to automatically tune the performance-sensitive configuration parameters. Through the use of N-Gram and Recurrent Neural Network (RNN) language modeling, we validate the intuition that the quality of the correction can be computed quantitatively and efficiently using the ``perplexity'' metric, prevalent in the NLP domain. After training the language model, we show that the perplexity metric calculated for runtime data has a strong negative correlation with the correction of the erroneous NGS reads. Therefore, we use the perplexity metric to guide a hill climbing-based search, converging toward the best $k$-value. Our approach is suitable for both \textit{de novo} and comparative sequencing (resequencing), eliminating the need for a reference genome to serve as the ground truth. This is important because use of a reference genome often carries forward the biases along the stages of the pipeline. Further, in more specialized applications such as variant-calling, a reference genome may in fact be misleading, making \name suitable for such applications. %\SCcomment{So, this last line shows why our method is useful even in comparative seq and some other applications}
% Mus: Makes sense.
